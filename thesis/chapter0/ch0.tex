%!TEX output_directory = texaux
%!TEX spellcheck
%!TEX root = ../main.tex

\setlength{\abovedisplayskip}{20pt}
\setlength{\belowdisplayskip}{20pt}

\chapter{Wstęp}

System dialogowy (inaczej \textit{chatbot}) to program komputerowy umożliwiający porozumienie w języku naturalnym między człowiekiem a maszyną. Możliwość naturalnej rozmowy z komputerem była kiedyś utożsamiana ze sztuczną inteligencją. Alan Turing, w swojej słynnej publikacji \cite{turing}, pytał czy da się napisać program konwersacyjny, którego ludzki obserwator nie byłby w stanie odróżnić od człowieka. Eksperyment ten dzisiaj nosi miano testu Turinga. Jego obecną formą jest Nagroda Loebnera -- coroczny konkurs, w którym twórcy programów starają się oszukać sędziów. Dotychczas wręczano jedynie nagrodę dla najlepszego chatbota. Nagrody głównej, jednorazowo przyznawanej programowi nieodróżnialnemu od człowieka, nie zdobył jeszcze żaden bot.

Przejście testu Turinga nie jest jednak celem każdego systemu dialogowego. Dzisiejsze chatboty mogą mieć bardzo różne przeznaczenie. Obecna w systemie iOS Siri\footnote{\url{https://www.apple.com/ios/siri/}} stanowi interaktywną pomoc dla użytkownika. Zastosowanie wykrywania mowy umożliwia wydawanie urządzeniu poleceń głosowych w języku naturalnym. Robot będący częścią portalu Visit Wroclaw\footnote{\url{https://www.facebook.com/visitwro/}} pełni rolę informacji turystycznej. Bardzo popularny Cleverbot\footnote{\url{http://www.cleverbot.com/}} cały czas uczy się lepiej mówić wykorzystując historię przeprowadzanych dialogów. Jest to system ogólnego przeznaczenia, potrafiący rozmawiać na wiele tematów. W biznesie chatboty często pełnią rolę interaktywnych asystentów, pomagając znaleźć połączenia lotnicze, zarezerwować miejsca w restauracji, czy zapoznać się z ofertą firmy.

Techniczne aspekty konstrukcji chatbotów również mogą się drastycznie różnić. Napisany w 1995 roku system \textit{Artificial Linguistic Internet Computer Entity} (w~skrócie A.L.I.C.E. lub Alice) \cite{alice} formułuje odpowiedzi wykorzystując dopasowanie wzorca i wypełnianie luk w schematach. Alice wymaga dużej liczby ręcznie utworzonych reguł, co ją mocno ogranicza. Obecnie często eksperymentuje się z wykorzystaniem metod uczenia maszynowego do budowania botów dialogowych. Szczególnie popularne są sieci neuronowe, które w ostatnich latach znacznie zwiększyły nasze możliwości w dziedzinie przetwarzania języka naturalnego.

\section{Cel i zawartość pracy} \label{celpracy}

Celem tej pracy jest przedstawienie niektórych mechanizmów neuronowych znajdujących zastosowanie w systemach konwersacyjnych. Rozdział~\ref{rozdzial1} stanowi wstęp do tematu sieci neuronowych i wprowadzenie niezbędnej nomenklatury. Kolejny opisuje statystyczne modelowanie języka, które jest podstawą sieci pracujących na języku naturalnym. Następne dwa rozdziały są główną częścią pracy. Rozdział~\ref{rozdzial3} pokazuje architekturę generującą wypowiedzi w dialogu korzystając z pojedynczych słów. Zamieszczam tam także przykłady otrzymanych dialogów. Rozdział~\ref{rozdzial4} jest opisem modelu potrafiącego wyciągać z tekstu odpowiedzi na pytania. Zawiera on też moją próbę rozszerzenia oryginalnej funkcji algorytmu, wraz z wynikami. Do pracy dołączone są implementacje opisywanych metod w Pythonie oraz instrukcja powtórzenia eksperymentów. Ze względu na uniwersalność i dostępność danych w przykładach korzystam z języka angielskiego.

Podczas przygotowywania tego opracowania byłem członkiem reprezentacji Uniwersytetu Wrocławskiego na NIPS 2017 Conversational Intelligence Challenge\footnote{\url{http://convai.io/}}. Celem konkursu było zaprogramowanie chatbota rozmawiającego o zadanym na początku rozmowy krótkim artykule. Zaprezentowany przez UWr robot \texttt{poetwanna.be} zdobył w zawodach 1. miejsce ex aequo. Mój wkład obejmował między innymi obsługę trudniejszych pytań o fakty. Udział w konkursie był przyczyną powstania rozdziału~\ref{rozdzial4} tego opracowania, oraz motywacją do przeprowadzenia opisanego w~sekcji~\ref{negans} eksperymentu z negatywnymi odpowiedziami.
